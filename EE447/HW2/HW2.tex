\documentclass{oxmathproblems}
\usepackage{graphicx}
\usepackage{listings}
\graphicspath{{imagenes/}} 

\course{HW 2 Double Spend Problem}
\begin{document}
\vspace{-15mm}

Assume result of attcking between different time intervals are independent, we can view the attacking process as a Markov Process. We model the expected gain (expected reward - cost) as $E_{m,n}$, where $m$ is the length of the main (being attacked) chain, and $n$ is the length of the attacker's chain.

We define the following expected gains:
\begin{enumerate}
    \item If $m=k$, $E_{m,n}=0$, the transaction is confirmed, failed attack leads to no gains.
    \item If $n=k$, $E_{m,n}=100$, the attacker's chain first get confirmed with $k$ lengths.
    \item For other cases, we have $E_{m,n} = 0.49 \times E_{m+1,n} + 0.51 \times E_{m,n+1} - 1$, indicating the attacks fails for 49\% probability, where the main chain grows by one and succeeds for 51\% probability, where the attacker's chain grows by one. $-1$ indicates the cost of an extra unit of time using the computation power.
\end{enumerate}


Our goal is to find a $k$ where $E_{1,0} > 0$, i.e. starting from the initial state, the expected gain of attacks is larger than the costs.

We write a program to calculate the whole $E_{m,n}$ table. We find that $k=29$ gives the first negative result in $E_{1,0}$

\begin{lstlisting}[language=Python]
K = 29
E_table = []
for i in range(K+1):
    E_table.append([0 for j in range(K+1)])

for i in range(K+1):
    E_table[i][K] = 100

for i in range(K-1,-1,-1):
    for j in range(K-1,-1,-1):
        E_table[i][j] = - 1 + 0.51 * E_table[i][j+1] +\
                         0.49 * E_table[i+1][j]

for i in range(K):
    print(29-i, E_table[1+i][i])
\end{lstlisting}

The expected gain for every $K= 29$ is $-0.3282 W$

\end{document}